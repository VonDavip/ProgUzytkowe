\documentclass[a4paper, 12pt]{article}
\usepackage[utf8]{inputenc}
\usepackage{polski}
\usepackage[pdftex]{hyperref}
\usepackage{graphicx}
\usepackage{enumerate}
\usepackage{multirow}
\usepackage{amsmath} %pakiet matematyczny
\usepackage{amssymb} %pakiet dodatkowych symboli
\title{Kolokiwum z Latex'a}
\date{18 stycznia 2017}					%%DATA , AUTOR%%
\author{David Ruszkowski}

\begin{document}
\maketitle    						%%STRONA TUTU�OWA%%
\tableofcontents					%%SPIS TRE�CI%%




\section{Sie� neuronowa}			%%PODZIA� PDF-A NA SEKCJE...%%	
MIEJSCE NA TEKST
	\subsection{Typy sieci neuronowych} %%..PODSEKCJE%%
																		
\begin{enumerate}[A)]					%%NUMERACJA TYPU A), inne:(a),i),I),1,..)%%
\item Sieci jakie� tam
\item Sieci rekurencyjne 
\begin{description}					%%PUNKTOWANIE JAKIE CHCEMY-->[...]%%
\item[i.]sie� Hopfielda 
\item[ii.]maszyna Boltzmanna
\end{description}  
\end{enumerate}

	\subsection{Zastosowania}
		\subsubsection{Przyk�ady wykorzystania}			%%PODPODSEKCJA%%



\newpage
\section{Wzory matematyczne}

	\begin{equation}						%%equation- TRYB MATEMAYCZNY%%
	\sum_{n=1}^{\infty}aq^{n-1}= \frac{a}{1-q}. 
	\label{eq:rownanie1}		%%ODWO�ANIE-->zeby wywo�a� \ref{eq:r�wnianie}%%
	\end{equation}
	
	\begin{equation}
	g^{\epsilon,cd}_{r_{gran}}(u)\not =\{v\in TRN; \frac{|IND_{\epsilon}(u,v)|}{|A|}\geq r_{gran}and\ d(u) = d(v)\} 
	\label{eq:rownanie2}
	\end{equation}
	
	$$If|g^{\epsilon,cd}_{r_{gran}}(x_{i})|=n\ then,$$ 		%%$$KOLEJNY TRYB MATEMATYCZNY $$..$$%%
	\begin{equation}																														%%�EBY WYWO�A� TRYB MATEMATYCZNY W TEK�CIE WYSTARCZY POJEDY�CZO $..$%%
	g^{\epsilon,cd}_{r_{gran}}(x_{i})=\left(\begin{array}{cccc}
	a_{1} & a_{2} & \ldots &a_{m}(x_{1})\\ 
	a_{1} & a_{2} & \ldots &a_{m}(x_{1})\\ 
	\ldots & \ldots & \ldots & \ldots \\ 
	a_{1} & a_{2} & \ldots &a_{m}(x_{1})
	\end{array}\right)
	\label{eq:rownanie3}
	\end{equation}
	
	\paragraph{}Wro�skian				%%PARAGRAF - raczej zb�dne%%
	\begin{equation}
	F(f_{1},f_{2},\ldots,f_{n})=\left[\begin{array}{cccc}			%%przestrze� array-- podobne dzia�anie jak w tabelach. {ccc..}-liczba kolumn(c-oznacza wy�rodkowanie)%%
	f_{1}&f_{2}&\ldots&f_{n}\\
	f^{'}_{1}&f^{'}_{2}&\ldots&f^{'}_{n}\\
	\vdots & \vdots & \ddots & \vdots \\ 
	f^{n-1}_{1}&f^{n-1}_{2}&\ldots&f^{n-1}_{n}
	\end{array}\right]
	\label{eq:rownanie4}
	\end{equation}
	
	

	
\section{Tabele i rysunki}

\begin{table}[h]
\begin{tabular}{lccccccc}
Team & P & W & D & L & F & A & Pts\\
\hline
Manchester United & 6&4&0&2&10&5&12\\
Celtic & 6&4&0&2&10&5&12\\
Benfica & 6&4&0&2&10&5&12\\
FC Copenhagen & 6&4&0&2&10&5&12\\
\end{tabular}
\end{table}

\begin{figure}[!ht]
\vspace{0pt}
\includegraphics[scale=0.3]{Neuralnetwork.eps}
\includegraphics[scale=0.5]{Neuralnetwork1.eps}
\vspace{0pt}
\caption{Funkcja b��du dla liniowego Neurona}			%%tytu� tabeli, b�dz obrazu%%
\end{figure}




\newpage
\section{Bibliografia}

Prace [1, 2] przedstawiaj� sposoby wyci�gania wiedzy z macierzy DNA, po-
nadto b��d uczenia sieci neuronowej zobrazowany jest na rysunku 1

\begin{thebibliography}{4}				%%BIBLIGRAFIA%%
\bibitem{brown}
\bibitem{Eisen}
\bibitem{furey}
\bibitem{zadeh}
\end{thebibliography}

\end{document}